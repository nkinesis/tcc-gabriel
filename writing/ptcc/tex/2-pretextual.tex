% Seleciona o idioma do documento (conforme pacotes do babel)
%\selectlanguage{english}
\selectlanguage{brazil}

% Retira espaço extra obsoleto entre as frases.
\frenchspacing 

\newpage

% ==============================================
% ELEMENTOS PRÉ-TEXTUAIS
% ==============================================
\pretextual

% ----------------------------------------------
% Capa
% ----------------------------------------------
%\imprimircapa
% Capa personalizada sem o uso de \imprimircapa
\begin{capa} 
   \center
  {\ABNTEXsubsectionfont\large \imprimirinstituicao} \linebreak
  {\ABNTEXsubsectionfont\large DEPARTAMENTO DE CIÊNCIAS EXATAS E ENGENHARIAS} \linebreak
  {\ABNTEXsubsectionfont\large CURSO DE CIÊNCIA DA COMPUTAÇÃO} \linebreak
  {\ABNTEXsubsectionfont\large CAMPUS SANTA ROSA}
   \vfill
   \ABNTEXchapterfont\Large\bfseries{\MakeUppercase{\imprimirtitulo}}
   \vfill
   \begin{center}
   \ABNTEXchapterfont\large\bfseries\textsc{\MakeUppercase{\imprimirautor}}
   \end{center}
   \vfill
   \vspace*{5cm}
   \large\normalfont\MakeTextUppercase{\imprimirlocal} \\
   \large\normalfont\the\year
   \vspace*{1cm}
\end{capa}

% ----------------------------------------------
% Folha de rosto
% ----------------------------------------------
% folha de rosto personalizada sem uso de \imprimirfolhaderosto
\makeatletter

\renewcommand{\folhaderostocontent} {
    \begin{center}
      {\ABNTEXsubsectionfont\large \imprimirinstituicao} \linebreak
      {\ABNTEXsubsectionfont\large DEPARTAMENTO DE CIÊNCIAS EXATAS E ENGENHARIAS} \linebreak
      {\ABNTEXsubsectionfont\large CURSO DE CIÊNCIA DA COMPUTAÇÃO} \linebreak
      {\ABNTEXsubsectionfont\large CAMPUS SANTA ROSA}
    \end{center}  
    \vspace*{3cm}
    \begin{center}
      \vspace*{\fill}
      \begin{center}
      \ABNTEXchapterfont\bfseries\Large\imprimirtitulo
      \end{center}
      \vspace*{\fill}
      \vspace*{3cm}
      \abntex@ifnotempty{\imprimirpreambulo}{%
        \hspace{.45\textwidth}
        \begin{minipage}{.5\textwidth}
        \SingleSpacing
        \imprimirpreambulo
        \end{minipage}
        \vspace*{\fill}
      }
    
      \abntex@ifnotempty{\imprimirorientador}{%
      \hspace{.45\textwidth}
      \begin{minipage}{.5\textwidth}
    	{\imprimirorientadorRotulo~\imprimirorientador}%
      \end{minipage}
      }
      
      \vspace*{3cm}
      
      \vspace*{\fill}
      {\large\imprimirlocal}
      \par
      {\large\the\year}
      \vspace*{1cm}
    \end{center}
}


\makeatother

% Folha de rosto
\folhaderostocontent


% ||||||||||||||||||||||||||||||||||||||||||||||
% RESUMOS
% ||||||||||||||||||||||||||||||||||||||||||||||

% ----------------------------------------------
% Resumo em português
% ----------------------------------------------
% Importante: De acordo com a NBR6024 as palavras-chaves devem ser separadas entre si por ponto e devem ter somente a primeira palavra escrita com letra maiúscula
%\iffalse
%\setlength{\absparsep}{18pt} % ajusta o espaçamento dos parágrafos do resumo
% \begin{resumo}
% 	
%     \lipsum[7]
%     
% 	\vspace{\onelineskip}
%  
% 	\noindent 
% 	\textbf{Palavras-chaves}: Palavras
% \end{resumo}

% ----------------------------------------------
% inserir lista de ilustrações
% ----------------------------------------------
% \pdfbookmark[0]{\listfigurename}{lof}
% \listoffigures*
% \cleardoublepage

% Diferentes tipos de listas podem ser criadas por meio de macros do memoir.

% ----------------------------------------------
% inserir lista de tabelas
% ----------------------------------------------
% \pdfbookmark[0]{\listtablename}{lot}
% \listoftables*
% \cleardoublepage

% ----------------------------------------------
% inserir lista de abreviaturas e siglas
% ----------------------------------------------
% Importante: As abreviaturas e siglas devem estar em ordem alfabética
% \begin{siglas}
%   \item[ABNT] Associação Brasileira de Normas Técnicas
%   \item[abnTeX] ABsurdas Normas para TeX
% \end{siglas}
% \fi

% ----------------------------------------------
% Página em branco
% ----------------------------------------------
% \cleardoublepage {\char"2800}
% \cleardoublepage {\char"2800}

% ----------------------------------------------
% inserir o sumário
% ----------------------------------------------
\pdfbookmark[0]{\contentsname}{toc}
\tableofcontents*
\thispagestyle{empty}
\pagestyle{simple}
%\thispagestyle{empty}
% \cleardoublepage
